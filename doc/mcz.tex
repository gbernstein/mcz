% Fixes to make:
% *a_zcB is conditioned on s already, need some notation for n(c,~s)?
% * Use a prior from neighbors for some redshfit outlier rates?
% * p_\alpha(B) = p_c(B) in redshifts leads to nothing?  3sDir?
% * Sys(z^\prime) = Sys_r instead?


% Paper on BFD implementation and testing
\documentclass[11pt,preprint,flushrt]{aastex631}

% For arXiv: \pdfoutput=1
%\documentclass{emulateapj}
\renewcommand{\topfraction}{0.8}
\usepackage{natbib,graphicx,amsmath,subfigure,color}
%\usepackage[draft]{todonotes}
\def\eqq#1{Equation~(\ref{#1})}
\newcommand\etal{{\it et al.\/}}
%\newcommand\etal{{\it et al. }}
\newcommand\eg{{\it e.g.\/}}
\newcommand\ie{{\it i.e.\/}}
\newcommand\etc{{\it etc.\/}}
% Vectors: bold italic.  Matrices: bold roman.  Components: not bold
\newcommand{\veca}{\mbox{\boldmath $a$}}
\newcommand{\vecs}{\mbox{\boldmath $s$}}
\newcommand{\vece}{\mbox{\boldmath $e$}}
\newcommand{\vecd}{\mbox{\boldmath $d$}}
\newcommand{\vecD}{\mbox{\boldmath $D$}}
\newcommand{\vecF}{\mbox{\boldmath $F$}}
\newcommand{\vecX}{\mbox{\boldmath $X$}}
\newcommand{\vecg}{\mbox{\boldmath $g$}}
\newcommand{\vecG}{\mbox{\boldmath $G$}}
\newcommand{\vecM}{\mbox{\boldmath $M$}}
\newcommand{\vecN}{\mbox{\boldmath $N$}}
\newcommand{\vecQ}{\mbox{\boldmath $Q$}}
\newcommand{\vecZ}{\mbox{\boldmath $Z$}}
\newcommand{\vect}{\mbox{\boldmath $t$}}
\newcommand{\vecx}{\mbox{\boldmath $x$}}
\newcommand{\veck}{\mbox{\boldmath $k$}}
\newcommand{\vecv}{\mbox{\boldmath $v$}}
\newcommand{\vecu}{\mbox{\boldmath $u$}}
\newcommand{\vecalpha}{\mbox{\boldmath $\alpha$}}
\newcommand{\vectheta}{\mbox{\boldmath $\theta$}}
\newcommand{\matR}{\mbox{$\bf R$}}
\newcommand{\matC}{\mbox{$\bf C$}}
\newcommand{\matB}{\mbox{$\bf B$}}
\newcommand{\matA}{\mbox{$\bf A$}}
\newcommand{\matI}{\mbox{$\bf I$}}
\newcommand{\bnab}{\boldsymbol{\nabla}}
\newcommand{\bnabg}{\boldsymbol{\nabla_g}}
\newcommand{\Sg}{\mbox{${\bf S}_g$}}
\newcommand{\likeli}{\mbox{$\mathcal{L}$}}
\newcommand{\real}{\operatorname{Re}}
\newcommand{\imag}{\operatorname{Im}}
\newcommand{\photoz}{photo-$z$}
%\newcommand\edit[1]{\textcolor{red}{#1}}
%\newcommand\edit[1]{#1}
\newcommand\answer[1]{\textcolor{violet}{#1}}
\begin{document}

%\slugcomment{early draft}

%\keywords{gravitational lensing: weak---methods: data analysis}
\title{$n(z)$ posterior sampling notes}

\author{Gary\ldots}
\email{garyb@physics.upenn.edu}
\affil{Department of Physics \& Astronomy, University of Pennsylvania, 
209 S.\ 33rd St., Philadelphia, PA 19104}

%\begin{abstract}
%We prescribe a method to construct a likelihood of obtaining a
%\end{abstract}

\section{Model definition}
Our goal is to sample from the space of possible redshift distributions $n(z | B)$ giving the sky
density $dn/dz$ for a ``wide'' sample of galaxies divided into bins
$B$.  These functions will be subject to constraints from several
observational resources:
\begin{itemize}
\item   A \textbf{deep} sample consists of galaxies with sufficient
  photometric information to be definitively assigned to one of a set
  of \textbf{phenotypes} indexed by $c,$ which in our application
  correspond to cells in a self-organized map (SOM) created from
  multi-band fluxes (``deep cells'').
\item A \textbf{spectroscopic} sample, which are sources having
  sufficient photometric information for definitive assignment to a
  deep cell $c,$ and \emph{also} sufficient additional data $D$ to
  derive a reliable probability $p(z | D)$ of the redshift.  This
  might be $p(z|D) = \delta(z-z_{\rm spec})$ for a secure
  spectroscopic redshift, or a finite-width or multimodal probability
  for multiband photometric observations.  The spectroscopic sample
  may overlap with the deep sample, but we will \emph{not} assume that
  the spectroscopic sample is representative of the deep (or any
  other) sample.
\item The \textbf{wide} sample, covering our entire usable survey,
  whose redshift distribution we wish to constrain.  Each wide galaxy
  is assigned to a unique category $\hat c$ (a ``wide cell'') based on
  its observed fluxes.  In our application, these are cells in another
  SOM based on the survey bands.  Groups of the wide cells are merged
  to form a smaller number of \emph{bins} indexed by $B.$
  \textcolor{red}{We also need to designate one wide ``cell'' $\hat
    c_{\bar s}$ for galaxies that are drawn from the deep sample but
    are either undetected, or are detected but do not pass selection
    cuts and are hence not in any of the wide SOM cells.  We can merge
    these unusable galaxies with those in any wide cell $\hat c$ that
    we choose to \emph{not} use in cosmological analyses into a common
    bin $B_T$, the ``trash bin,'' (a.k.a.\ ``compost'' or ``garbage''
    bin).  When we sum over bins $B$ we will include the compost bin,
    unless noted otherwise.}
\item The cross-correlation or \textbf{WF} data measure the angular
  cross-correlation $w_{Br}$ between the wide-survey members of some
  bin $B$ and a set of reference samples $r$ having well-constrained $n(z).$
\end{itemize}

The underlying idea of our SOMPZ process is that we are reweighting
the spectroscopic sample to match the wide sample, as (first?)
proposed by Lima et al. (2008).  In any reweighting process, 
we must first decide what set of
characteristics $x$ of a galaxy suffice to insure that all galaxies
satisfying conditions $x$ are drawn from the same $n(z)$.  If this is true we
can posit that any spectroscopic measurement of a galaxy in $x$ is
drawn from $p(z|x),$ and hence its $p(z|D)$ is an unbiased sample from $p(z|x).$
Making the conditions $x$ more restrictive makes it more likely that
we are isolating galaxies with identical $p(z)$'s---but slicing the
spectroscopic sample into smaller groups leaves us with weaker
constraints on each $p(z|x).$    On the other hand, a coarser set of
$x$ categories runs the risk that our spectroscopic sample is not
truly a representative sample of $x.$

The finest $x$ division would be to let each source in the deep field
be its own $x,$ but we clearly don't have enough spectroscopy to use
this.  A too-coarse division might be to use the wide-field bins, $x=B$.
In between these two are, in order of increasing specificity,
$x=c,$ $x=(c,B),$ and $x=(c,\hat c).$  In DES Y3, simulations showed
that $x=c$ was potentially too coarse, and using $x=(c,B)$ was as good
as using $x=(c,\hat c).$   We will move ahead with this ``bin
conditionalization'' approach.

Our task, therefore, is to sample from the possible functions
$n(z|c,B)=p(z|c,B) n(c,B)$ that are consistent with the four sources
of data listed above.

We will model the $n(z)$ for any particular sample as a linear sum
\begin{equation}
  n(z) = \sum_i a_i K_i(z),
\end{equation}
where $a_i$ are unknown coefficients of predetermined functions $K_i.$
In DES Y3, the $K_i$ were chosen to be boxcar functions of equal width
$\Delta z$:
\begin{align}
  K_i(z) & = \Pi\left(\frac{z}{\Delta z}-i\right), \\
  \Pi(u) & = \begin{cases}
    1 & 0<u<1 \\
    0 & \text{otherwise.}
  \end{cases}
\end{align}
This gives the piecewise-constant $n(z)$ that has problems because it
does not have $n(z)\rightarrow 0$ as $z\rightarrow0,$ and is also ugly
because it's discontinuous.  We can instead construct a
piecewise-linear $n(z)$ using
\begin{align}
  K_i(z) & = \Lambda\left(\frac{z}{\Delta z}-i\right), \\
  \Lambda(u) & = \begin{cases}
    u & 0<u<1 \\
    2-u & 1<u<2 \\
    0 & \text{otherwise.}
  \end{cases}
\end{align}
Both the boxcar and sawtooth functions have the nice feature that
$n(z)$ is non-negative if and only if the $a_i$ are non-negative.  A
smoother $n(z),$ which is quadratic as $z\rightarrow 0$ like the real
universe should be, and would have continuous first derivatives, would
use
\begin{align}
  K_i(z) & = Q\left(\frac{z}{\Delta z}-i\right), \\
  Q(u) & = \begin{cases}
    u^2/2 & 0<u<1 \\
    3/4 - \left(u-3/2\right)^2 & 1<u<2 \\
    (3-u)^2/2  & 2<u<3 \\
    0 & \text{otherwise.}
  \end{cases}
\end{align}
With the quadratic kernel, it is still true that $\{a_i\}\ge 0
\Rightarrow n(z) \ge 0.$  But the converse is not true---there are
non-negative $n(z)$ functions that can be created with some $a_i<0.$

In any case we can proceed with the model that
\begin{equation}
  n(z^\prime | c, B) = \sum_z a_{zcB} K_z(z^\prime),
  \label{azcB}
\end{equation}
where we now use $z$ as a discrete index over the coefficients.  Our
technique will be to derive the posterior distribution on the
parameter vector
$\veca=\{a_{zcB}\}$ conditional on the observations.  Then from each sample,
\begin{equation}
  n(z^\prime | B) = \sum_c a_{zcB}  K_z(z^\prime).
\end{equation}

\section{Constraints}
The posterior distribution of $\veca$ is
\begin{equation}
  p(\veca | D) \propto Pr(\veca) \prod_i p(D_i | \veca),
  \label{posterior}
\end{equation}
where $Pr$ is a prior and $D=\{D_w, D_d, D_s,D_{wz}\}$ is the
amalgamation of the observations of wide, deep, and spectroscopic
samples, and the WZ data.  We can now derive these likelihood terms
under the product.

\subsection{Deep counts}
The deep galaxies can each be assigned a $c$ value, leading to counts
$M_c$ of such galaxies in the deep survey area $A_d$.  These are assumed to be randomly drawn from a Cox process,
which is a Poisson distribution modulated by some fluctuations
$\Delta^d_z$ in the galaxy density under each redshift kernel $K_z$:
\begin{align}
M_c & \sim {\rm Poisson}\left[A_d \sum_{z,B} a_{zcB} (1+\Delta^d_z)\right] \\
\Rightarrow \quad p(D_d | \veca, \{\Delta^d_z\})& \propto
  Pr(\{\Delta_z^d\})
  \exp\left[-A_d \sum_{z,B} a_{zcB} (1+\Delta^d_z)\right]
  \prod_c  \left[\frac{A_d}{M_c!} \sum_{z,B} a_{zcB} (1+\Delta^d_z)\right]^{M_c}
  \\
\nonumber
  \Rightarrow \quad -\log p(D_d | \veca, \{\Delta^d_z\}) & = A_d \sum_{zcB}  a_{zcB}
                                          (1+\Delta^d_z) - \sum_c M_c
                                           \log\left[\sum_{z,B}
                                           a_{zcB}
                                           (1+\Delta^d_z)\right] \\
   & \phantom{=} - \log  Pr(\{\Delta_z^d\}) + 
                                           \text{const}.
\label{deeplike}
\end{align}
As in the 3sDir sampling method by S\'anchez \etal\ (2020), a
covariance matrix for the $\Delta$ values can be derived from a
nominal power spectrum for the mass fluctuations.  Since 
each redshift shell of the deep fields is large, $O(100\,{\rm Mpc}),$
we can assume a normal distribution for the overdensities.  There will
be some bias factor $b^2$ between the computed mass covariance and the
galaxy covariance matrix $C_\Delta;$ we could allow this to be a free
parameter, or even assign different biases to different subsects of
$c$, but for now we will hold it constant, so that
\begin{align}
  \{\Delta^d_z\} & \sim {\mathcal N}(0, C_d) \\
  \Rightarrow \quad -\log Pr(\Delta^d) & = \frac{1}{2}
                                          \{\Delta^d_z\}^T C_d^{-1}
                                          \{\Delta^d_z\}  +
                                          \text{const}.
\end{align}

\subsection{Wide counts}
Each wide galaxy is assigned to some cell $\hat c$ and hence we can
accumulate the counts $M_{\hat c}$ in each wide cell.  A model for
these requires that we know the \textbf{transfer function} $f_{\hat
  c}^c = p(\hat c | c)$ giving the probability that a galaxy in deep
cell $c$ will be placed in $\hat c$ after being observed with the
noise levels in the wide survey.  Note that this function will depend
on observing conditions, but to start with we will assume that we have
the values area-averaged over the wide footprint, in which case it
will be true that
\begin{equation}
  \langle M_{\hat c} \rangle = A_w \sum_{cz} f^c_{\hat c} a_{zcB}.
\end{equation}
There is no summation over $B$ since it is determine by $\hat
c.$\footnote{\textcolor{red}
{The deselection ``cell''  $\hat c_{\bar s}$ has an $f^c_{\hat c}$
  too, but we do not have an observable number of counts for it so it
  cannot be used for constraints.  It might be useful in this case to
  have two compost bins, one for the undetected/unselected sources,
  and one for galaxies in wide cells $\hat c$ which we measure but do
  not use.  This might offer a more informative constraint.}}  We
could add the mean density fluctuations to the expected density as we
did for the deep counts, but since it is the distribution of
the realization of galaxies in the wide field that we want for our
analyses, not a universal average, these factors should properly be
absorbed into the $a_{zcB}.$

Furthermore the area of the wide survey is large enough that the
Poisson fluctuations in the wide-cell densities and the density fluctuations
will be small---there are $\approx 10^8$ wide galaxies distributed
over $\approx 10^3$ wide SOM cells.  Instead of treating the wide
counts as a probability distribution, we could consider this a fixed
constraint:
\begin{equation}
 M_{\hat c} = A_w \sum_{cz} f^c_{\hat c} a_{zcB}.
\label{wideconstraint}
\end{equation}
The values of the $f^c_{\hat c}$ are taken to be exactly known because
each deep galaxy has been injected into the wide survey and assigned
to a $\hat c$ many times in the Balrog process, making the statistical
errors in the transfer function well below statistical uncertainties
from other observational constraints.  The total selection rate
$\sum_{\hat c} f^c_{\hat c}$ can be anywhere between 0 and 1 for a
given $c$.

\subsection{Spectroscopic data}
Each spectroscopic measurement $\alpha$ is considered an independent random
draw from $p(z|c_\alpha,B)$ for the deep cell assignment of galaxy
$\alpha,$ conditional on the bin assignment.  The spectroscopic or multi-band analysis of the data
$D_\alpha$ for that source yield a distribution $p(z_\alpha |
D_\alpha),$ and since such estimates are nearly always made with a
uniform prior on $z_\alpha,$ we have $p(D_\alpha | z_\alpha) \propto p(z_\alpha |
D_\alpha)$ and we can use them interchangeably.

We also will need to know the probability $f^\alpha_B \equiv p_\alpha(B)$ that deep
galaxy $\alpha$ would end up in bin $B$ when it appears in the wide
survey.  These values\textcolor{red}{will sum to unity if we include
  the compost bin.}
\begin{align}
  p(D_\alpha |\veca, \{\Delta^s_z\}) & = \int dz_\alpha p(D_\alpha| z_\alpha)
                           p(z_\alpha | \veca, \{\Delta^s_z\}) \\
  & \propto \int dz_\alpha p(z_\alpha| D_\alpha) \frac{ \sum_B
    n(z_\alpha|c_\alpha,B) p_\alpha(B)} { \int dz^\prime \sum_B  n(z^\prime |
    c_\alpha,B) p_\alpha(B)} \\
  & = \frac{ \sum_{Bz} f^\alpha_z a_{zc_\alpha B} (1+\Delta^s_z)
    f^\alpha_B}{\sum_{Bz} a_{zc_\alpha B} (1+\Delta^s_z) f^\alpha_B}, \\
  f^\alpha_z & \equiv \int dz_\alpha p(z_\alpha | D_\alpha)
               K_z(z^\prime).
\end{align}
The last line assumes that we have normalized our $n(z)$ kernel
elements such that $\int dz K_i(z)=1.$  We have introduced galaxy
density fluctuations $\Delta^s_z$ averaged over the spectroscopic
survey area as nuisance parameters, and $f_z^\alpha$ encapsulates the
redshift measurement results for galaxy $\alpha.$

If we again approximate the joint distribution of the $\Delta^s_z$ as
a multivariate normal with covariance matrix $C_s,$ then the
likelihood term for the spectroscopic data becomes
\begin{align}
  \nonumber
- \log  p(D_s | \veca, s, \{\Delta^s_z\}) & = -\sum_\alpha \log\left[  \sum_{Bz}
                        f^\alpha_z a_{zc_\alpha B} (1+\Delta^s_z)  f^\alpha_B
                        \right]  +\sum_\alpha \log\left[\sum_{Bz} a_{zc_\alpha B}
     (1+\Delta^s_z) f^\alpha_B\right] \\
     & \phantom{=} +  \frac{1}{2}
                                         \{\Delta^s_z\}^T C_s^{-1}
                                         \{\Delta^s_z\}  +
                                         \text{const}.
       \label{speclike}
\end{align}

If the spectroscopic and deeps field regions overlap, there will be
covariance between $\Delta^s$ and $\Delta^d,$ so we will need to make
a joint prior with a covariance matrix $C_{ds}$ that subsumes the
$C_s$ and $C_d$ terms at the ends of Equations~(\ref{deeplike}) and (\ref{speclike}).

\subsection{Clustering likelihood}
Following Gatti et al. (2022) from Y3,  the observed
cross-correlation coefficients $w_{Br}$ between galaxies in bin $B$
and reference samples indexed by $r$ centered at redshifts $z_r$ is
modeled as
\begin{align}
\label{wzmodel}
  \hat w_{Br} & = \int dz^\prime f(z^\prime|B) p_r(z^\prime) b_r w_{\rm
                DM}(z^\prime) {\rm Sys}(z^\prime, \vecs_B) \\
\nonumber 
   & \phantom{=} + \alpha_r \int_0^{z_r} dz^\prime f(z^\prime|B)
     b_B(z^\prime) D(z^\prime, z_r) \\
\nonumber 
   & \phantom{=} + b_r \int_{z_r}^\infty dz^\prime f(z^\prime|B)
     \alpha_B(z^\prime) D(z_r,z^\prime) \\
  f(z^\prime|B) & \equiv \frac{n(z^\prime|B)}{\int
    dz^{\prime\prime} n(z^{\prime\prime} | B)} = \frac{ \sum_c
                  a_{zcB}K_z(z^{\prime})}{\sum_c a_{zcB}}
                  \label{fzB}
\end{align}
The first line of the expression is a simple scale-independent,
linear-bias model for the expected angular clustering, multiplied by a
mild function of $z$ parameterized by some values of $\vecs$ that are
left free to vary.  This ${\rm Sys}(z)$ function is meant to absorb
the failures of the linear-bias model and our uncertainties in the
redshift dependence of the bias of galaxies in $B$, and a prior on
$\vecs$ expresses our estimate of the size and redshift variability of
these corrections to the model.  The clustering of
matter, $w_{\rm DM},$ is estimated from a reference cosmological
model---any deviation of the true cosmology from this reference
cosmology is absorbed into the Sys function.

The second line models angular correlations
caused by magnification of reference galaxies at $z_b$ by mass
associated with source galaxies in bin $B$ at redshift $z^\prime.$
The third line is the term from magnification of the lensing sources at
$z^\prime>z_b.$  The function $D(z_{\rm lens},z_{\rm src})$ includes
cosmological and geometric factors along with the power spectrum of
matter fluctuations.  We have assumed in these terms that $n_r(z)$ is
narrow and centered at $z_r,$ with bias $b_r$ and magnification
coefficient $\alpha_r$ being constant over this range.  For the
sources, we have at least a free bias parameter $b_B$ and
magnification coefficient $\alpha_B,$ that most generally might be
functions of $z.$  Image simulations will give Gaussian priors on the
$\alpha$ values.

Under the (reasonable) assumption that the Sys function varies weakly
across any given redshift kernel $K_z(z^\prime),$ and similarly 
we can replace the functions $b_B(z^\prime), \alpha_B(z^\prime),$ and
$D(z^\prime,r)$ with constants $b_{Bz}, \alpha_{Bz},$ and $D_{zr},$
then can remove all the $z^\prime$ integrations in (\ref{wzmodel})
with summations over our coefficients.
\begin{align}
\label{wzsum}
  \hat w_{Br} & = \frac{1}{\sum_c a_{zcB}} \sum_{zc} a_{zcB} \left[  W_{zr}\, {\rm Sys}(z,\vecs_B) 
 + \alpha_r b_{Bz} D_{zr} + \alpha_{Bz} b_r D_{rz}\right] \\
W_{zr} & \equiv \int dz^\prime K_z(z^\prime) p_r(z^\prime) b_r w_{\rm
         DM}(z^\prime).
         \label{Wzr}
\end{align}
In the above we have absorbed into $D_{zr}$ the condition that $z<z_r$
by setting the $z>z_r$ elements to zero.

The likelihood $p(D_{wz} | \veca)$ becomes
\begin{align} 
\label{wzsum}
-\log p(D_{wz} | \veca, \ldots) & = \frac{1}{2} ({\bf w} - \hat{\bf w})^T (C^{wz})^{-1}
                          ({\bf w} - \hat{\bf w}) + \text{const} \\
  \nonumber & \phantom{=} -\log Pr(\{\alpha_r\}) - \log
              Pr(\{\alpha_{Bz}\}) -  \log Pr(\{b_r\}) - \log
              Pr(\{b_{Bz}\})  \\
  \nonumber & \phantom{=} -  \sum_B \log Pr(\vecs_B) .
\end{align}
where we have known covariance for the measurements ${\bf w} =
\{w_{Br}\},$ and the model $\hat {\bf w}$ is dependent on the
$a_{zcB}$ parameters and the WZ nuisance parameters as in
Equations~(\ref{wzsum}) and (\ref{Wzr}).  We note that the log posterior is
quadratic in the bias and magnification values $\alpha_{Bz}$ and
$b_{Bz}$, and also in the $\vecs_B$ if the Sys function is linear in
  them and they have a Gaussian prior.  Hence we could marginalize
  over these analytically (for fixed $\veca, \alpha_r,$ and $b_r$)
  rather than sampling over them.

\section{Summary}
The \textbf{data} provided to the calculation are:
\begin{itemize}
  \item The deep counts in SOM cells $M_c$ and the area $A_d$ with
    which they are associated.
  \item The wide counts in SOM cells $M_{\hat c}$ and the area $A_w$
    with which they are associated.
  \item The spectroscopic likelihoods $p(z_\alpha | D_\alpha)$ and
    deep-cell assignments $c_\alpha$ of all the targets of the
    spectroscopic sample.
  \item The WZ angular correlations $w_{Br}$ against reference
    samples, and the covariance matrix $C^{wz}$ for them.
  \end{itemize}

The \textbf{free parameters} of the model are:
\begin{itemize}
  \item The conditional sky densities $\veca = \{a_{zcB}\}$.  If there
    are 4 bins, $64\times64=4096$ deep SOM cells, and $\approx50$
    redshift bins, this is a maximum of 800k parameters.  We expect,
    however, for this matrix to be sparse, in that a given $c$ might
    produce sources $<4$ bins, and be constrained to be non-zero in
    only a small fraction of the redshift bins, leaving $<1e5$ free
    parameters to sample.  The wide-SOM occupancy constraint in
    Equation~(\ref{wideconstraint}) will project away $\approx10^3$ of
    them.
  \item The remaining parameters are nuisances.  First are the
    overdensities $\Delta^d$ and $\Delta^s$ of the deep and
    spectroscopic fields, $\approx100$ total free parameters.
  \item The bias factors $b_{B}$ and Sys-function coefficients
    $\vecs_B$ will be $\approx 5$ free parameters per bin, or $\approx
    20$ total, along with magnification coefficients $\alpha_r$ for
    each reference set ($\approx 100$) and source bin (4) are free in
    the model $\hat w_{zr}.$
\end{itemize}
  
The \textbf{calculated inputs} to the model are:
\begin{itemize}
  \item The transfer function matrix $f^c_{\hat c}$ derived from
    Balrog injections or other means.
  \item The bin probabilities $p_\alpha(B)$ of each spectroscopic
    galaxy's assignment into bins, again estimated from Balrog
    injections of that source.
  \item The covariance matrix $C^{ds}$ of the $\Delta^s$ and
    $\Delta^d$ overdensities, for their joint multivariate normal
    distribution. 
  \item The effective areas $A_d$ and $A_s$ of the deep and
    spectroscopic surveys.
  \item The dark matter clustering $w_{DM}(z)$ and prefactors
    $D(z,z^\prime)$ for magnification signals in a nominal
    cosmological model.
  \item Means and variances for Gaussian priors on $\alpha_r$ and $\alpha_B$ derived from
    magnification tests in Balrog or other means.
  \item The biases $b_r$ derived from autocorrelations of the
    reference populations via the nominal cosmology.
  \item Means on the source biases $b_B$ and the variances for
    Gaussian priors on these and each individual $\vecs_B$ element.
\end{itemize}
  
In addition we must posit a \textbf{prior on \veca}, included which
elements to fix to zero.  This prior will control the $n(z)$ assigned
to cells that are deficient in spectroscopic data, and will control
the uncertainties ascribed to rare redshift outliers within deep
cells.

The total posterior probability on $\veca$ and the implied $n(z|B)$
functions is then defined by this prior on 
$\veca,$ along with the data likelihoods given in
Equations~(\ref{deeplike}),
(\ref{speclike}), and (\ref{wzlike}), with the contraint
(\ref{wideconstraint}).  The individual terms of the likelihood are
algebraically simple to calculate and differentiate, and highly
amenable to pipelined calculations.  The challenge is
to efficiently sample the large $\veca$ space using these derivatives.


\section{Questions}
\begin{enumerate}
  \item In equation 5 the sawtooth function appears to have a potential
    typo. Should it be $2-u$ when $1<u<2$?

    \answer{Yes! Fixed.}

\item Regarding equation 12, I’m curious why $\Delta^d_z$ isn’t integrated
over. My understanding is that this prior might require
integration. However, if integrated, it seems equation 13 couldn’t be
easily derived…

For the Deep count, since Deltas are normal distributions, shouldn't the equations 12-13 be marginalised over it ?

\answer{Correct, the final probability will need to be marginalized
  over the $\Delta$ values.  My assumption is that the $\Delta$ values are all nuisance
  parameters that will be sampled along with the $\veca$ values.  I
  have noted now that these probabilities are also conditional on
  $\Delta$.}

\answer{The $A_d$ would correspond to the total area of the sky over
  which the deep galaxies are collected.  The calculation of the
  Gaussian prior on $\Delta^d_z$ would need to account for the
  geometry of these fields.  The $A_s, \Delta^s_z$ are more complicated because the
  spectra come from a variety of geometries; I'm ignoring that for now.}

\item In the same section, do we have only one deep survey ? (and one Ad)

  As a clarification, $A_w$ and $A_d$ are effective area that takes number
density into account. By equation 16, does this imply a reweighting of
deep galaxies to wide without altering the total galaxy count?

\answer{I'm not sure what you mean by ``takes number density into
  account.'' The $a$ values will be in units of galaxies per unit
  area.  They represent a mean value over density fluctuations.  It
  may be true that $\sum_{\hat c} f^c_{\hat c} < 1$ if the galaxies in
  $c$ are not all detected in the wide survey, so the total wide
  counts do not add up to total deep counts in a fixed area.}

\item minor comment, eq 10 should be summed over z as well right ?
  
\answer{Do you mean a different equation number??}


\item For the Wide counts, the transfer function take into account the
selection right? hence the use of Balrog. (so if I understand
rigorously eq 16-17 it is $M_c|s$ and f is conditionalised by s as well
.? )

\answer{Since a galaxy that is assigned to a $\hat c$ is already
  selected, the counts $M_{\hat c}$ are already conditioned on
  selection, and the $f^c_{\hat c}$ are too.}

\item Eq 16-17, shouldn't it be $A_w/A_d$ as a prefactor ? Maybe this
  depends on the definition of f...

\answer{As noted above, the $a$ values are in units of galaxies per
  unit area, so to predict the galaxy counts we will multiply by the
  appropriate $A$ for each survey component.}
  
\item Is the spectroscopic data section detailed in some of your articles ?
  I don't understand eq 18--21

  \answer{I don't know if we wrote anything like this in the SOMPZ
    papers.  I see that there are some mistakes here though (which
    I've fixed above).  The $p(D_\alpha, z_\alpha)$ should be
    $p(D_\alpha | z_\alpha)$ which is $\propto p(z_\alpha | D_\alpha)$
    because the people analyzing the spectro sample have assumed a
    uniform prior on $z$ when estimating their probabilities.}

\answer{Also I have put in now the dependence of $p(z_\alpha)$ on the
  $\veca$ that are determining $n(z)$.  At eqn (19), note that
  conditioning on $s$ is the same as summing over $B$ because a
  selected source must end up in one of the bins $B$.}

\item Why is selection effect important for the spectroscopic part, but not
for the Deep counts ? Many  deep galaxies will not be selected in the
wide right ?

\answer{For the deep counts, there is no selection, there is no
  selection; we are assuming that we are getting every galaxy in the
  deep fields which has any chance of being selected in the wide
  survey.  When we do the spectro fields, we are conditioning on the
  bin membership (which implies a selection).}

\item Eq 18, shouldn't it be $p(D|z)$ ?

  \answer{Yes---I think I fixed the place this should be?}

\item why is p(z|s) not conditionalised by a in 18 ? (is there a physical
  reason to drop conditionalisation for $p(D|z)$ ?)

  \answer{Yes, it is conditioned in $a$, I put this in.}

\item I don't understand how you go to line 19, with the c variable,
  and why it becomes $c_\alpha$ in 20

  \answer{Should be a $c_\alpha$ in line (19) too---I am assuming here
    that we know exactly which deep-SOM bin every spectro galaxy
    belongs to.}
  
\end{enumerate}

\end{document}
                  